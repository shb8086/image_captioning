%%%%%%%%%%%%%%%%%%%%%%%%%%%%%%%%%%%%%%%%%%%%%%%%%%%%
%    Canadian AI Latex Template    %
%%%%%%%%%%%%%%%%%%%%%%%%%%%%%%%%%%%%%%%%%%%%%%%%%%%%

\documentclass[10pt]{cai}

% Graphics should all go in the figs/ directory
\graphicspath{{figs/}}


\begin{document}

% Editorial staff will replace the following values:
% 1. Conference Year
% 2. Issue number
% 3. Article DOI
\volumeheader{36}{0}%{00.000}
\begin{center}

  \title{Image Captioning Dataset in Persian Using Flicker30k Images}
  \maketitle

  \thispagestyle{empty}

  % Add Authors and Affiliations in the camera ready
  % for the double blind review, please leave this section as is 
  \begin{tabular}{cc}
    Shima Baniadamdizaj\upstairs{\affilone}, Alexander Breuer\upstairs{\affilone}
   \\[0.25ex]
   {\small \upstairs{\affilone} Fredrich Schiller University Jena} \\
  \end{tabular}
  
  % Replace with corresponding author email address
  \emails{
    \upstairs{*}shima.bani@uni-jena.de
    }
  \vspace*{0.2in}
\end{center}

\begin{abstract}
  Image Captioning is the task of describing an image with a textual description in natural language. 
  It makes use of each Natural Language Processing and Computer Vision to generate the captions. 
  For this task, it is necessary to identify objects, actions, their relationship, and some silent feature that may be missing in the image.
  The final result must be a syntactically and semantically correct generated, relevant and brief description of the image.
  The use cases are computer vision and natural language processing (NLP). 
  It is not an easy task for a machine to generate or imitate the human brain's ability for image captioning, although research in this field has shown great achievements. 
  Deep learning techniques are capable to handle such problems using CNN and LSTM. There are many datasets in this case, but mostly contain captions in English, whereas datasets with captions described in other languages are scarce. 
  In this paper, we introduce a new and at the moment first image captioning dataset in Persian. 
  The images are collected from the Flickr30k dataset and have five different references per image. 
  Also, both the mean and variance of reference sentence length are high, which makes this dataset challenging due to its linguistic aspect. \\
\end{abstract}

% add your keywords
\begin{keywords}{Keywords:}
  Image Captioning, Image Captioning Dataset, Image-To-Text, Computer Vision, Natural Language Processing
\end{keywords}
\copyrightnotice

\section{Introduction}
\label{intro}
We invite papers that present original work in all areas of Artificial Intelligence, either theoretical or applied. Canadian AI 2023 welcomes submissions on topics including (but not limited to):

\
We expressly encourage work that cuts across technical areas or applies AI techniques in the context of important domains such as e-commerce, games, healthcare, sustainability,  transportation, Internet of Things, and agriculture.

Moreover, we plan to have a collocated event on Ethics of AI, with a joint invited keynote and other activities related to this topic. We therefore encourage papers highlighting research in this specific area.

We also welcome the submission of position papers, which present evidence-based arguments for a particular point of view without necessarily presenting a new system. There will be an option during the submission process to indicate that a paper is a position paper.

\subsection{Important Dates}
\label{important}

\begin{table}[h]
\begin{tabular}{ |l|l| }
 Submission deadline: & February 12th, 2023 (11:59 p.m. UTC-12) \\ 
 Author notification: & March 26th, 2023 \\  
 Final papers due: & April 9th, 2023 \\
 Main conference: & June 5 to 9, 2023
\end{tabular}
\vspace{0.2cm}
\caption{Important dates}
\label{tab:important}
\end{table}

\section{Submission Details}
\label{submission}
We invite submissions of both long and short papers. Long papers must be no longer than \textbf{12 pages}, and Short papers must be no longer than \textbf{6 pages}, including references, formatted using the conference template. The authors should consult the authors guidelines and use the provided proceedings template to prepare their papers  \cite{cai2020,author1_name_author_2020}.

Papers submitted to the conference must not have already been published, or accepted for publication, or be under review by a journal or another conference. Submissions will go through a \textbf{double-blind} review process by Program Committee members to assess originality, significance, technical merit, and clarity of presentation. As such, submissions must be anonymized, and papers that fail to do so will be rejected without review. A “Best Paper Award” and a “Best Student Paper Award” will be given at the conference respectively to the authors of each best paper, as judged by the Best Paper Award Selection Committee.

\section{Publication}
\label{pub}

The conference proceedings will be published in PubPub open access online format \cite{pubpub2020}, and submitted to be indexed/abstracted in leading indexing services such as DBLP, ACM, Google Scholar. A paper will be accepted either as a long or as a short paper. Long papers will be allocated 12 pages while short papers will be allocated 6 pages in the proceedings. Authors of accepted papers will be allocated time for an oral presentation at the conference and will have the opportunity to present their work in a poster session. 

At least one author of each accepted paper is required to attend the conference to present the work. The authors must agree to this requirement prior to submitting their paper for review.

In addition, the corresponding author of each paper, acting on behalf of all of the authors of that paper, must complete and sign a Consent-to-Publish form. The corresponding author signing the copyright form should match the corresponding author marked on the paper.

\section*{Acknowledgements}
Canadian AI is sponsored by the Canadian Artificial Intelligence Association(CAIAC)\cite{caiac}.

%Begin appendix section(s)
\appendix

% Add appendices here:
\section{Example of math equation }
%\label{appendix-customize-this-label}
Binomial theorem: \cite{abramowitz1948handbook}
\begin{equation}
(x+y)^n=\sum_{\substack{k=0}}^{n}\dbinom{n}{k}x^{n-k}y^k
\end{equation}




% All references should be stored in the file "references.bib"
% Please do not modify anything below this line.
%\thispagestyle{plain}
%\printbibliography

\printbibliography[heading=subbibintoc]

\end{document}
